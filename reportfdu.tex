% !TEX TS-program = xelatex
% !TEX encoding = UTF-8 Unicode
%                                  ---by suchot

\documentclass[AutoFakeBold]{fdureport}

%\usepackage{graphicx}
%\usepackage{subfigure}
\begin{document}
%=====%
%
%封皮页填写内容
%
%=====%

% 标题样式 使用 \title{{}}; 使用时必须保证至少两个外侧括号
%  如: 短标题 \title{{第一行}},  
% 	      长标题 \title{{第一行}{第二行}}
%             超长标题\tiitle{{第一行}{...}{第N行}}

\title{{工作报告}}



% 标题样式 使用 \entitle{{}}; 使用时必须保证至少两个外侧括号
%  如: 短标题 \entitle{{First row}},  
% 	      长标题 \entitle{{First row}{ Second row}}
%             超长标题\entitle{{First row}{...}{ Next N row}}
% 注意:  英文标题多行时 需要在开头加个空格 防止摘要标题处英语单词粘连。
\entitle{{Report}}

\author{似然}
\major{应用数学}
\advisor{似然}
\college{类脑智能科学与技术学院}
\grade{2020级}



\maketitle
\frontmatter

%英文摘要
%\EnAbstract{Fudan University was established in 1905 as Fudan Public School. It was the first institution of higher education to be founded by a Chinese person. The two characters, fù (“return”) and dàn (“dawn”) were borrowed from A Commentary on The Classic of History, of which the part on the Yu and the Xia dynasties mentions: “Brilliant are the sunshine and moonlight, again the morning radiance returns at dawn.” In 1917, the institution was renamed Fudan University, which has been kept ever since.  
%}{FDU, key university,outstanding}
%中文摘要
%\ZhAbstract{复旦大学,位于上海市,是中华人民共和国教育部直属、中央直管副部级建制的全国重点大学,世界一流大学建设高校(A类),国家“985工程”、“211工程”重点建设高校,入选国家“珠峰计划”、“强基计划”、“111计划”、“2011计划”、卓越医生教育培养计划、卓越法律人才教育培养计划、国家建设高水平大学公派研究生项目、新工科研究与实践项目、中国政府奖学金来华留学生接收院校、全国首批深化创新创业教育改革示范高校、首批学位授权自主审核单位,九校联盟、环太平洋大学联盟、中国大学校长联谊会、东亚研究型大学协会、新工科教育国际联盟、医学“双一流”建设联盟、长三角研究型大学联盟、 长三角高校智库联盟创始成员,中国大学智库论坛秘书处单位,是一所国内顶尖的综合性研究型大学。
%}{复旦大学,重点大学,双一流}
%=======%
%插入引言
%=======%



%生成目录
\addtocontents{toc}{\protect\thispagestyle{empty}}
\tableofcontents
%插入图或表的目录

\renewcommand\listfigurename{插\ 图\ 目\ 录}
\renewcommand\listtablename{表\ 格\ 目\ 录}
\addtocontents{lof}{\protect\thispagestyle{empty}}
\listoffigures
%\thispagestyle{empty}
%\newpage
\addtocontents{lot}{\protect\thispagestyle{empty}}
\listoftables
%\thispagestyle{empty}
%\newpage
%文章主体
\mainmatter

\Intro
这里是工作总述部分


\chapter{研究背景}
\section{二级标题}
\subsection{三级标题}
\par \textbf{章节引用}示例,参见python代码见附录\ref{sec:code}
\par 听说\textbf{数学公式}输入好南

\begin{equation}
\begin{aligned}
\mathrm{KL}(p \| q) &=-\int p(\boldsymbol{x}) \ln q(\boldsymbol{x}) \mathrm{d} \boldsymbol{x}-\left(-\int p(\boldsymbol{x}) \ln p(\boldsymbol{x}) \mathrm{d} \boldsymbol{x}\right) \\
&=-\int p(\boldsymbol{x}) \ln \left\{\frac{q(\boldsymbol{x})}{p(\boldsymbol{x})}\right\} \mathrm{d} \boldsymbol{x}
\end{aligned}
\label{eq:KL}
\end{equation}

\par 引用KL散度公式,见公式\ref{eq:KL}。

\par 对于公式输入,我选择使用OCR识别,推荐一个小工具\href{https://accounts.mathpix.com/signup?referral_code=8ZTrrDwYv3}{Mathpix}(点击文档中的\href{https://accounts.mathpix.com/signup?referral_code=8ZTrrDwYv3}{Mathpix})

\par 听说\textbf{表格输入}也不太简单。学位论文通常使用三线表
\begin{table}[htbp] 
	\centering	
	\begin{tabular}{lcl} 
		\toprule 
		性别 & 身高 & 体重 \\ 
		\midrule 
		 女 & 165 & 60 \\ 
	     男 & 175 & 70 \\ 
		 女 & 161 & 55 \\ 
		\bottomrule 
	\end{tabular} 
\caption{\label{tab:test}示例表格} 
\end{table}
\par 听说你想\textbf{并排插入图像}
\begin{figure}[H]
	\begin{minipage}[t]{0.5\textwidth}
	\centering
	\tiny 图片1 \\
	\vspace{0.5cm}
	\includegraphics[scale=0.05]{anime.jpg}
\end{minipage}
\begin{minipage}[t]{0.5\textwidth}
	\centering 
	\tiny 图片2 \\
	\vspace{0.5cm}
	\includegraphics[scale=0.05]{anime.jpg}  
\end{minipage}	
\protect\caption{两张图并排 \label {fig:two-pics}}	
\end{figure}

\par 重要的事强调三遍,详细内容见ReadMe(求你,一定要看) \\
{\bfseries 编译方式:} XeLaTeX -->BibTeX --> XeLaTeX-->XeLaTeX \\
{\bfseries 编译方式:} XeLaTeX -->BibTeX --> XeLaTeX-->XeLaTeX \\
{\bfseries 编译方式:} XeLaTeX -->BibTeX --> XeLaTeX-->XeLaTeX
\chapter{研究方法}
\par 再多我也不会了,请学会正确使用搜索引擎。
\par 我适配的附录代码是python,可修改language选项适配其他语言。

\begin{lstlisting}[language = python]
print("hello world!")

\end{lstlisting}

\chapter{研究结果}

\chapter{总结与讨论}

%论文后部
\backmatter


%=======%
%引入参考文献文件
%=======%
\bibdatabase{bib/database}%bib文件名称 仅修改bib/ 后部分
\printbib




\Appendix
\section{python代码}
\label{sec:code}


这里是附录页,附上你的程序或必要的相关知识\cite{partl2016}

{\bfseries 编译方式:} XeLaTeX -->BibTeX --> XeLaTeX-->XeLaTeX



\end{document}